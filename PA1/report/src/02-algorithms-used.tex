\section{Algorithms Used}\label{section:algorithms-used}
To retrieve the minimum weight dominating set of a graph $G$, two different types of algorithm were used.
The first algorithm was an exhaustive search algorithm, while the rest followed a greedy heuristics.

\subsection{Exhaustive Search}\label{section:exhaustive-search}

Exhaustive search is a brute-force approach to combinatorial problems that consists of generating every element of the problem domain, verify if it fulfils a specific condition and then finding a desired element \cite{levitin2012introduction}.

The exhaustive search used for the problem at hand is based on generating every possible combination of edges of the graph.
Then, for every single combination, verify it is a dominating set.
If so, compute the sum of weights of every edge that belongs to the combination.
Then, compare that sum with the current minimum weight.
If the sum is smaller than the minimum, the combination is a better solution than the current minimum set.
Therefore, the combination at hand is the now the current best solution and the sum of the weights is the current minimum weight.
This algorithm is better illustrated in \autoref{alg:exhaustive}.

\begin{algorithm}
\caption{Exhaustive search algorithm}
\label{alg:exhaustive}
\begin{algorithmic}
\Inputs{$G(V,E) \gets$ graph with set of vertices $V$ and set of edges $E$}
\Initialize{$l_c \gets$ list of every combination of edges in $E$

$w_{min} \gets \infty$

$set_{min} \gets [\cdot]$}

\For{$c$ in $l_c$}
    \If{$c$ is dominating set of $G(V,E)$}

        $w_c = \sum$ weight of edges in $c$
        
        \If{$w_c < w_{min}$}

           \State $w_{min} \gets w_c$
           \State $set_{min} \gets c$

        \EndIf        
    \EndIf

\EndFor

\end{algorithmic}
\end{algorithm}

\subsection{Greedy Heuristics}
A greedy approach is a general design technique applicable to optimization problems.
It consists a solution through a set of steps, expanding a partially constructed solution along them.
On each step, a choice need to take place.
That choice needs to be: feasible, so it satisfies the problem's constraints; locally optimal, i.e., it has to be the best possible choice among all feasible choices available; and irreversible, i.e., the choice, once made, cannot be changed \cite{levitin2012introduction}.

For the current problem, three different greedy heuristics were developed: minimum weight, maximum connection and one based on the work of Chaurasia and Singh\cite{chaurasia}.

In the first one, edges of the graph are sorted in ascending order by weight. 
Then, the edge with the least weight is added to the solution set.
The solution set is checked to verify if it is a dominating set.
If so, the solution is found and the algorithm stops.
If not, add the next edge to the solution edge.
The algorithm is described in \autoref{alg:min-weight}. 

\begin{algorithm}
\caption{Minimum weight greedy heuristics}
\label{alg:min-weight}
\begin{algorithmic}
\Inputs{$G(V,E) \gets$ graph with set of vertices $V$ and set of edges $E$}
\Initialize{$l_{E,W} \gets$ list of $E$ with corresponding set of weights $W$

$w_{min} \gets 0$

$set_{min} \gets [\cdot]$}


\State $l_{E,W-sorted} = l_{E,W}$ sorted in ascending order by weight
\For{$edge,\ weight$ in $l_{E,W-sorted}$}

    \State $set_{min} \gets$ add $edge$
    \State $w_{min} += weight$
    \If{$set_{min}$ is dominating set of $G(V,E)$}

        \State \textbf{break}
    \EndIf
\EndFor
\end{algorithmic}
\end{algorithm}

The second one is similar to the first, with the difference lying in the sorting step.
Instead of sorting in ascending order by weight, this heuristics sorts the edges in descending order by number of adjacent edges, as seen in \autoref{alg:max-connection}.

\begin{algorithm}
\caption{Maximum connection greedy heuristics}
\label{alg:max-connection}
\begin{algorithmic}
\Inputs{$G(V,E) \gets$ graph with set of vertices $V$ and set of edges $E$}
\Initialize{$l_{E,W,NA} \gets$ list of $E$ with corresponding sets of weights $W$ and of the number of adjacent edges $NA$

$w_{min} \gets 0$

$set_{min} \gets [\cdot]$}


\State $l_{E,W,NA-sorted} = l_{E,W,NA}$ sorted in descending order by number of adjacent edges
\For{$edge,\ weight,\ n_{adjacent}$ in $l_{E,W,NA-sorted}$}

    \State $set_{min} \gets$ add $edge$
    \State $w_{min} += weight$
    \If{$set_{min}$ is dominating set of $G(V,E)$}

        \State \textbf{break}
    \EndIf
\EndFor
\end{algorithmic}
\end{algorithm}

% \newpage
The third greedy algorithm is based on the work by Chaurasia-Singh\cite{chaurasia}.
In this algorithm, a weight ratio for a certain edge is calculated by dividing the sum of the weights of its adjacent edges with its weight.
The edges are then sorted in descending order by their weight ratio.
The algorithm is exemplified in \autoref{alg:chaurasia}.

\newpage

\begin{algorithm}
\caption{Chaurasia-Singh greedy heuristics}
\label{alg:chaurasia}
\begin{algorithmic}
\Inputs{$G(V,E) \gets$ graph with set of vertices $V$ and set of edges $E$}
\Initialize{$l_{E,W,A,W_{A}} \gets$ list of $E$ with corresponding sets of weights $W$, of adjacent edges $A$ and of weight of adjacent edges $W_A$}

$w_{min} \gets 0$

$set_{min} \gets [\cdot]$

$l_{W_{r} \gets [\cdot]}$

\For{$edge,\ weight,\ edges_{adjacents}, weight_{adjacents}$ in $l_{E,W,A,W_{A}}$}
    
    $W_{ratio} = weight_{adjacents} / weight$
    \State $l_{W_{r}} \gets$ add $W_{ratio}$

\EndFor

\State $l_{E,W,A,W_{A},W_{r}} \gets [l_{E,W,A,W_{A}},\ l_{W_{r}}]$

\State $l_{E,W,A,W_{A},W_{r}-sorted} = l_{E,W,A, W_{A}, W_{r}}$ sorted in descending order by weight ratio $w_r$
\For{$edge,\ weight,\ edges_{adjacent},\ weight_{adjacent}, weight_{ratio}$ in $l_{E-W-Asorted}$}

    \State $set_{min} \gets$ add $edge$
    \State $w_{min} += weight$
    \If{$set_{min}$ is dominating set of $G(V,E)$}

        \State \textbf{break}
    \EndIf
\EndFor
\end{algorithmic}
\end{algorithm}