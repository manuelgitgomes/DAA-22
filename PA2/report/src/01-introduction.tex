\section{Introduction}\label{section:introduction}

The need for gathering statistics on a large number of events is a very common occurrence in data science applications, such as social media networks and search engines.
However, the resources are finite and the totality of data to analyse is larger than the available memory.
The largest challenge in this area is to store the data in a storage efficient way. %TODO citation
One solution for this problem is the usage of probabilist counter algorithms.
In these algorithms, an occurrence is not always accounted, being dependent on a probabilty.
Although this reduces the memory used, the accuracy is reduced as well.


The objective behind this report is to explore the accuracy and efficiency behind some probabilistic counters in counting the occurrence of words inside a book.
The report is divided in five section: the first (\autoref{section:introduction}), where the problem is introduced;
the second (\autoref{section:algorithms-used}), where the algorithms used are described;
the third (\autoref{section:results}), where the experiments are described and results for them are detailed;
and the fourth (\autoref{section:conclusions}), where conclusions are extracted.
