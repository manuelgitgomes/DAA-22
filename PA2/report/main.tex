%
% General structure for the revdetua class:
%
\documentclass[shortpaper,english,final]{revdetua}
%
% Valid options are:
%
%   longpaper --------- \part and \tableofcontents defined
%   shortpaper -------- \part and \tableofcontents not defined (default)
%
%   english ----------- main language is English (default)
%   portugues --------- main language is Portuguese
%
%   draft ------------- draft version
%   final ------------- final version (default)
%
%   times ------------- use times (postscript) fonts for text
%
%   mirror ------------ prints a mirror image of the paper (with dvips)
%
%   visiblelabels ----- \SL, \SN, \SP, \EL, \EN, etc. defined
%   invisiblelabels --- \SL, \SN, \SP, \EL, \EN, etc. not defined (default)
%
% Note: the final version should use the times fonts
% Note: the really final version should also use the mirror option
%

\usepackage{amsmath,hyperref}
\usepackage{algorithm}
\usepackage{algpseudocode}
\usepackage{graphicx}
\usepackage{multirow}
\usepackage{titlesec}
\algnewcommand{\Inputs}[1]{%
  \State \textbf{Inputs:}
  \Statex \hspace*{\algorithmicindent}\parbox[t]{.8\linewidth}{\raggedright #1}
}
\algnewcommand{\Initialize}[1]{%
  \State \textbf{Initialize:}
  \Statex \hspace*{\algorithmicindent}\parbox[t]{.8\linewidth}{\raggedright #1}
}
\newcommand{\algorithmautorefname}{Algorithm}

\titleclass{\section}{straight}

\begin{document}

\Header{1}{1}{Janeiro}{2023}{1}
% Note: the month must be in Portuguese

\title{On the usage of approximate counters for the occurrences of words}
\author{Manuel Gomes, 88939} % or \author{... \and ...}
\maketitle

\begin{abstract}% Note: in English
This report present three counter algorithms to solve a problem consisting of counting the number of words inside different editions of the same book.
The first one is an exact while the rest are based on probabilistic counters.
After their implementation in Python 3, an experimental analysis of each algorithm is described.
\end{abstract}

\begin{resumo}% Note: in Portuguese
Este relatório apresenta três algoritmos de contadores para resolver um problema consistente em contar o número de palavras dentro de diferentes edições do mesmo livro.
O primeiro é um algoritmo exato, enquato que os restantes são baseados em contadores probabilísticos.
Após a sua implementação em Python 3, uma análise experimental de cada algoritmo é descrita.
\end{resumo}


% \begin{keywords}% Note: in English (optional)
%   ...
% \end{keywords}

% \begin{palavraschave}% Note: in Portuguese (optional)
%   ...
% \end{palavraschave}


\section{Introduction}\label{section:introduction}
Consider a finite, undirected graph $G(V,E)$ , where $V$ is the set of graph vertices and $E$ is the set of graph edges.
Two vertices $u, v$ of $G$ connected by edge $(u,v)$ are called adjacent nodes.
Two edges which share a vertex are also called adjacent.
One edge \emph{dominates} its adjacent.
A set of edges $M$ of $G$ is called an \emph{edge dominating set} if all edges of set $E - M$ are adjacent, and thus dominated, by the edges of $M$ \cite{dominating}.
The weight of an edge dominating set is the sum of its edges' weight. 
A \emph{minimum weight edge dominating set} is an edge dominating set whose total weight is as small as possible.

The objective behind this report is to apply exhaustive search and greedy algorithms to retrieve the minimum weight edge dominating set for a general graph $G$.
The report is divided in five section: the first (\autoref{section:introduction}), where the problem is introduced;
the second (\autoref{section:algorithms-used}), where the algorithms used are described;
the third (\autoref{section:formal-analysis}), where a formal analysis for the algorithms is presented;
the fourth (\autoref{section:results}), where results for experiments conducted with the algorithms are detailed;
and the fifth (\autoref{section:conclusions}), where conclusions are extracted.
\section{Algorithms Used}\label{section:algorithms-used}
\section{Results}\label{section:results}
After implementing the algorithms described in \autoref{section:algorithms-used}, experimental results were retrieved.
These results range from top-20 most frequent words to statistics for the top-3 words.
The results were taken in a machine with the AMD Ryzen 5 5600X processor and implemented in Python 3.
The experiments were taken using the book \emph{Manifesto of the Communist Party}, by Karl Marx and Friedrich Engels, due to its historical importance, various number of editions and availability of various editions for free-use. %TODO add citations
The three editions used will be described from now on as \emph{2004 edition}, \emph{2005 edition}, and \emph{2010 edition}. 
Every edition had every punctuation mark removed, as well the most common stop-words. %TODO add citation
Each edition had one thousand runs of the approximate counters algorithms, in order to retrieve a vast number of results.
The fixed probability counter used a probability of $\frac{1}{4}$, while the Csűrös' counter used a $d$ of $3$.

\subsection{2004 edition}

For the 2004 edition, the top-20 words calculated by the exact counter, an average of one thousand fixed probability counters, and an average of one thousand Csűrös' probability counters are shown in \autoref{fig:2004-20-exact}, \autoref{fig:2004-20-fixed}, and \autoref{fig:2004-20-csuros}, respectively.
From these graphs we can observe that the top-6 words are the same in every case. 
The first discrepancy seen is on the sixth position of \autoref{fig:2004-20-fixed}, where \emph{production} overtakes \emph{conditions}.
Both approximate counters present only four different relative positions in the graphs.
Regarding the same words present in the top-20, the Csűrös' counter graph present the word \emph{development} in favour of the word \emph{communists}.


\begin{figure}[!ht]
    \centering
    \includegraphics[width=0.9\linewidth]{figs/2004.epub-total}
    \caption{2004 edition's top-20 words according to the exact counter.}
    \label{fig:2004-20-exact}
\end{figure}


\begin{figure}[!ht]
    \centering
    \includegraphics[width=0.9\linewidth]{figs/2004.epub-fixed-1000}
    \caption{2004 edition's top-20 words according to the average of one thousand fixed probability counters.}
    \label{fig:2004-20-fixed}
\end{figure}


\begin{figure}[!ht]
    \centering
    \includegraphics[width=0.9\linewidth]{figs/2004.epub-csuros-1000}
    \caption{2004 edition's top-20 words according to the average of one thousand Csűrös' counters.}
    \label{fig:2004-20-csuros}
\end{figure}


\autoref{tab:2004} presents the statistics of the top-3 words for the 2004 edition.
In this statistics we can observe that the difference between the expected and the mean counter values is slim.
It is also verifiable that the fixed probability counter present a larger standard deviation than the Csűrös's counter.
The mean relative error is smaller in the Csűrös' counter, while the mean accuracy ratio stands closer to the number one on the fixed probability counter.


\begin{table}[!ht]
    \centering
    \caption{Statistics of the top-3 words for the 2004 edition, according to the fixed probability counter (Fixed) and the Csűrös' Counter (Csűrös).}
    \label{tab:2004}
    \resizebox{\columnwidth}{!}{%
    \begin{tabular}{l|cc|cc|cc|}
    Word                   & \multicolumn{2}{c|}{Class}          & \multicolumn{2}{c|}{Bourgeois}      & \multicolumn{2}{c|}{Bourgeoisie}    \\ \hline
    Counter Type           & \multicolumn{1}{c|}{Fixed} & Csűrös & \multicolumn{1}{c|}{Fixed} & Csűrös & \multicolumn{1}{c|}{Fixed} & Csűrös \\ \hline
    Real Value             & \multicolumn{2}{c|}{137}            & \multicolumn{2}{c|}{110}            & \multicolumn{2}{c|}{99}             \\ \hline
    Expected Counter Value & \multicolumn{1}{c|}{34.25} & 33.06  & \multicolumn{1}{c|}{27.5}  & 30.75  & \multicolumn{1}{c|}{24.75} & 29.38  \\ \hline
    Mean Counter Value     & \multicolumn{1}{c|}{34.19} & 32.78  & \multicolumn{1}{c|}{27.68} & 30.41  & \multicolumn{1}{c|}{24.72} & 29.2   \\ \hline
    Standard Deviation     & \multicolumn{1}{c|}{4.96}  & 2.24   & \multicolumn{1}{c|}{4.52}  & 2.26   & \multicolumn{1}{c|}{4.25}  & 2.44   \\ \hline
    Maximum Absolute Error & \multicolumn{1}{c|}{18.75} & 9.06   & \multicolumn{1}{c|}{15.5}  & 6.75   & \multicolumn{1}{c|}{12.25} & 8.62   \\ \hline
    Mean Absolute Error    & \multicolumn{1}{c|}{3.95}  & 1.72   & \multicolumn{1}{c|}{3.675} & 1.88   & \multicolumn{1}{c|}{3.41}  & 2.03   \\ \hline
    Mean Relative Error    & \multicolumn{1}{c|}{0.115} & 0.051  & \multicolumn{1}{c|}{0.134} & 0.061  & \multicolumn{1}{c|}{0.138} & 0.069  \\ \hline
    Mean Accuracy Ratio    & \multicolumn{1}{c|}{0.998} & 0.991  & \multicolumn{1}{c|}{1.006} & 0.989  & \multicolumn{1}{c|}{0.999} & 0.994  \\ \hline
    Smallest Counter Value & \multicolumn{1}{c|}{19}    & 24     & \multicolumn{1}{c|}{15}    & 24     & \multicolumn{1}{c|}{14}    & 24     \\ \hline
    Largest Counter Value  & \multicolumn{1}{c|}{53}    & 40     & \multicolumn{1}{c|}{43}    & 37     & \multicolumn{1}{c|}{37}    & 38    
    \end{tabular}%
    }
    \end{table}
  
    
Additional data retrieved during the experience by using the \verb|sys.getsizeof()| function states that the exact counter takes 147568 bytes of memory, while the approximate counters only take on average 40 bytes.

\newpage
\subsection{2005 edition}

For the 2005 edition, the top-20 words calculated by the exact counter, an average of one thousand fixed probability counters, and an average of one thousand Csűrös' probability counters are shown in \autoref{fig:2005-20-exact}, \autoref{fig:2005-20-fixed}, and \autoref{fig:2005-20-csuros}, respectively.
From these graphs we can observe that the top-9 words are the same in every case. 
The first discrepancy seen is on the sixth position of \autoref{fig:2005-20-fixed} and \autoref{fig:2005-20-csuros}, where \emph{production} overtakes \emph{conditions}.
Both approximate counters present only four different relative positions in the graphs.
Every graph shares the same top-20 words.


\begin{figure}[!ht]
    \centering
    \includegraphics[width=0.9\linewidth]{figs/2005.epub-total}
    \caption{2005 edition's top-20 words according to the exact counter.}
    \label{fig:2005-20-exact}
\end{figure}


\begin{figure}[!ht]
    \centering
    \includegraphics[width=0.9\linewidth]{figs/2005.epub-fixed-1000}
    \caption{2005 edition's top-20 words according to the average of one thousand fixed probability counters.}
    \label{fig:2005-20-fixed}
\end{figure}


\begin{figure}[!ht]
    \centering
    \includegraphics[width=0.9\linewidth]{figs/2005.epub-csuros-1000}
    \caption{2005 edition's top-20 words according to the average of one thousand Csűrös' counters.}
    \label{fig:2005-20-csuros}
\end{figure}


\autoref{tab:2005} presents the statistics of the top-3 words for the 2010 edition.
In this statistics we can observe that the difference between the expected and the mean counter values is slim.
Similarly to \autoref{tab:2004}, the fixed probability counter present a larger standard deviation than the Csűrös's counter and the mean relative error is smaller in the Csűrös' counter.
The mean accuracy ratio is similar between counters.


\begin{table}[!ht]
    \centering
    \caption{Statistics of the top-3 words for the 2005 edition, according to the fixed probability counter and the Csűrös' Counter.}
    \label{tab:2005}
    \resizebox{\columnwidth}{!}{%
    \begin{tabular}{l|cc|cc|cc|}
    Word                   & \multicolumn{2}{c|}{Bourgeois}       & \multicolumn{2}{c|}{Bourgeoisie}     & \multicolumn{2}{c|}{Class}          \\ \hline
    Counter Type           & \multicolumn{1}{c|}{Fixed}  & Csűrös & \multicolumn{1}{c|}{Fixed}  & Csűrös & \multicolumn{1}{c|}{Fixed} & Csűrös \\ \hline
    Real Value             & \multicolumn{2}{c|}{97}              & \multicolumn{2}{c|}{91}              & \multicolumn{2}{c|}{90}             \\ \hline
    Expected Counter Value & \multicolumn{1}{c|}{24.25}  & 29.12  & \multicolumn{1}{c|}{22.75}  & 28.38  & \multicolumn{1}{c|}{22.50} & 28.25  \\ \hline
    Mean Counter Value     & \multicolumn{1}{c|}{24.23}  & 28.98  & \multicolumn{1}{c|}{22.621} & 28.281 & \multicolumn{1}{c|}{22.49} & 28.2   \\ \hline
    Standard Deviation     & \multicolumn{1}{c|}{4.34}   & 2.36   & \multicolumn{1}{c|}{3.93}   & 2.27   & \multicolumn{1}{c|}{4.14}  & 2.23   \\ \hline
    Maximum Absolute Error & \multicolumn{1}{c|}{13.75}  & 6.88   & \multicolumn{1}{c|}{13.25}  & 6.62   & \multicolumn{1}{c|}{15.50} & 6.25   \\ \hline
    Mean Absolute Error    & \multicolumn{1}{c|}{3.466}  & 1.96   & \multicolumn{1}{c|}{3.14}   & 1.87   & \multicolumn{1}{c|}{3.31}  & 1.81   \\ \hline
    Mean Relative Error    & \multicolumn{1}{c|}{0.1429} & 0.067  & \multicolumn{1}{c|}{0.138}  & 0.066  & \multicolumn{1}{c|}{0.147} & 0.641  \\ \hline
    Mean Accuracy Ratio    & \multicolumn{1}{c|}{0.999}  & 0.995  & \multicolumn{1}{c|}{0.994}  & 0.997  & \multicolumn{1}{c|}{1.000} & 0.998  \\ \hline
    Smallest Counter Value & \multicolumn{1}{c|}{12}     & 23     & \multicolumn{1}{c|}{11}     & 23     & \multicolumn{1}{c|}{11}    & 22     \\ \hline
    Largest Counter Value  & \multicolumn{1}{c|}{38}     & 36     & \multicolumn{1}{c|}{36}     & 35     & \multicolumn{1}{c|}{38}    & 34    
    \end{tabular}%
    }
\end{table}


Similarly to the previous edition, additional data retrieved during the experience by using the \verb|sys.getsizeof()| function states that the exact counter takes 147568 bytes of memory, while the approximate counters only take on average 40 bytes.

\newpage
\subsection{2010 edition}

For the 2010 edition, the top-20 words calculated by the exact counter, an average of one thousand fixed probability counters, and an average of one thousand Csűrös' probability counters are shown in \autoref{fig:2010-20-exact}, \autoref{fig:2010-20-fixed}, and \autoref{fig:2010-20-csuros}, respectively.
From these graphs we can verify that they all share the same words in the same relative position.

\begin{figure}[!ht]
    \centering
    \includegraphics[width=0.9\linewidth]{figs/2010.epub-total}
    \caption{2010 edition's top-20 words according to the exact counter.}
    \label{fig:2010-20-exact}
\end{figure}


\begin{figure}[!ht]
    \centering
    \includegraphics[width=0.9\linewidth]{figs/2010.epub-fixed-1000}
    \caption{2010 edition's top-20 words according to the average of one thousand fixed probability counters.}
    \label{fig:2010-20-fixed}
\end{figure}


\begin{figure}[!ht]
    \centering
    \includegraphics[width=0.9\linewidth]{figs/2010.epub-csuros-1000}
    \caption{2010 edition's top-20 words according to the average of one thousand Csűrös' counters.}
    \label{fig:2010-20-csuros}
\end{figure}


\autoref{tab:2010} presents the statistics of the top-3 words for the 2010 edition.
In this statistics we can observe that the difference between the expected and the mean counter values is slim, as seen in previous counters.
The observations taken for \autoref{tab:2005} can be taken here as well, from the larger standard deviation on the fixed probability counter to the smaller mean relative error in the Csűrös' counter.
\begin{table}[!ht]
\centering
\caption{Statistics of the top-3 words for the 2010 edition, according to the fixed probability counter and the Csűrös' Counter.}
\label{tab:2010}
\resizebox{\columnwidth}{!}{%
\begin{tabular}{l|cc|cc|cc|}
Word                   & \multicolumn{2}{c|}{Class}          & \multicolumn{2}{c|}{Bourgeois}      & \multicolumn{2}{c|}{Bourgeoisie}    \\ \hline
Counter Type           & \multicolumn{1}{c|}{Fixed} & Csűrös & \multicolumn{1}{c|}{Fixed} & Csűrös & \multicolumn{1}{c|}{Fixed} & Csűrös \\ \hline
Real Value             & \multicolumn{2}{c|}{135}            & \multicolumn{2}{c|}{99}             & \multicolumn{2}{c|}{92}             \\ \hline
Expected Counter Value & \multicolumn{1}{c|}{33.75} & 32.94  & \multicolumn{1}{c|}{24.75} & 29.38  & \multicolumn{1}{c|}{23.00} & 28.50  \\ \hline
Mean Counter Value     & \multicolumn{1}{c|}{33.77} & 32.69  & \multicolumn{1}{c|}{24.67} & 29.33  & \multicolumn{1}{c|}{22.79} & 28.38  \\ \hline
Standard Deviation     & \multicolumn{1}{c|}{4.97}  & 2.28   & \multicolumn{1}{c|}{4.37}  & 2.32   & \multicolumn{1}{c|}{4.14}  & 2.32   \\ \hline
Maximum Absolute Error & \multicolumn{1}{c|}{17.25} & 9.06   & \multicolumn{1}{c|}{15.25} & 7.62   & \multicolumn{1}{c|}{13.00} & 7.50   \\ \hline
Mean Absolute Error    & \multicolumn{1}{c|}{3.95}  & 1.70   & \multicolumn{1}{c|}{3.45}  & 1.91   & \multicolumn{1}{c|}{3.31}  & 1.90   \\ \hline
Mean Relative Error    & \multicolumn{1}{c|}{0.117} & 0.052  & \multicolumn{1}{c|}{0.140} & 0.065  & \multicolumn{1}{c|}{0.144} & 0.067  \\ \hline
Mean Accuracy Ratio    & \multicolumn{1}{c|}{1.000} & 0.992  & \multicolumn{1}{c|}{0.997} & 0.999  & \multicolumn{1}{c|}{0.991} & 0.996  \\ \hline
Smallest Counter Value & \multicolumn{1}{c|}{18}    & 24     & \multicolumn{1}{c|}{12}    & 22     & \multicolumn{1}{c|}{11}    & 21     \\ \hline
Largest Counter Value  & \multicolumn{1}{c|}{51}    & 42     & \multicolumn{1}{c|}{40}    & 37     & \multicolumn{1}{c|}{36}    & 36    
\end{tabular}%
}
\end{table}

Similarly to the previous subsections, additional data retrieved during the experience by using the \verb|sys.getsizeof()| function states that the exact counter takes 147568 bytes of memory, while the approximate counters only take on average 40 bytes.

\newpage
\section{Conclusions}\label{section:conclusions}
In this report, three different algorithm were presented to count the number of words inside a book.
The first one is an exact algorithm, counting every occurrence of every word, prioritizing accuracy to efficient memory usage.
The second is a fixed probability counter, which reduces accuracy in order to occupy less memory.
This can however lead to large inaccuracies.
To tackle this, the third counter, a Csűrös' counter was implemented.
In this counter, the probability of counting a word decreases with the number of equal words counted. 

Results show that, on average, both probabilistic counters are accurate.
However, the fixed probability counter presents itself with a larger standard deviation.
This is due to the chaotic nature of the sampling, following only a fixed probability.
The Csűrös' counter has a lower standard deviation, which is to be expected, due to the probability of counting a word decreases with number of counts, leading to more severe probabilities for the top words.
This severe probability creates an environment where augmenting your counter does not happen very often, leading to counter values being less spread out.
This effect can also be visualized in the mean relative error, which is considerably smaller in the Csűrös' counter. 
Some can argue that the mean accuracy ratio is worse than the one presented by the fixed probability counter, which is true, but only shows that the average value of the fixed counter stays closer to the expected one, which does hold significant value when the variance is high.
It also needs to be taken into account that due to reasons previously explained the Csűrös' counter is \emph{low-biased}, achieving results that, on average, are smaller than the ones expected.
This can be visualized on the mean accuracy error, never surpassing 1.
Nevertheless, both of the probabilistic counters occupy much less memory than the exact one.

Results also show a difference between different editions of the same book, especially the ones retrieved from Project Gutenberg, where the words \emph{project} and \emph{gutenberg} appear in the top-10.

Possible work on this subject would be the implementation of more types of counters, such as the Morris' counter, and comparing them with the results obtain in here.
Another possible line of work would be to invest in larger computational experiments to better validate these results.

\bibliography{references.bib}
% \bibliography{...} % use a field named url or \url{} for URLs
% Note: the \bibliographystyle is set automatically

\end{document}
